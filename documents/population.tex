% This file is part of the Undetectable project.
% Copyright 2013 David W. Hogg (NYU)

\documentclass[12pt]{article}
\newcommand{\documentname}{\textsl{Note}}
\begin{document}

\begin{abstract}
Imagine an extremely ``faint'' or low-signal astronomical source, like
some kind of very tiny exoplanet or stellar oscillation.  Imagine that
there are many of these objects out there, but that not a single one
has ever been detected significantly in \emph{any} data set.  In this
\documentname, we ask the insane question ``Given observations of
enough systems, can we confidently infer properties of the population
of sources, even if not a single one is detected in any data set?''
The answer, of course, is ``yes'': So long as enough systems have been
observed such that the sum of the squares of all the individually low
signal-to-noise ratios (in all the individually observed systems) is
large, it is possible in principle to make confident statistical
statements about the population as a whole.  The method proposed here
involves hierarchical probabilistic inference.  It works well on toy
data---in this case artificial exoplanet radial-velocity data---but it
suffers from the problem that (almost by assumption) population
inferences are hard to test with existing or new data; while parameter
estimation and model comparison are possible, informative model
checking is nearly impossible.
\end{abstract}

\end{document}
